\documentclass[11pt]{report}
\usepackage[spanish]{babel}
\usepackage[utf8]{inputenc}
\usepackage[T1]{fontenc}
\usepackage{graphicx}




\begin{document}
\title{Estructura de la atmósfera}
\author{Gonzalez Escalante Luis Fernando
\and Profesor Lizarraga Carlos}

\date{29 Enero 2017}
\maketitle
\begin{abstract}
Mediante este documento se pretende informar un poco sobre la atmósfera y algunas propiedades.
Se describira un poco sobre la estructura y composición de ésta.

La atmósfera es parte importante de lo que hace posible que la Tierra sea habitable.Ya que bloquea y evita que agunos de los peligrosos rayos del Sol lleguen a la Tierra.
\end{abstract}
\section*{Atmósfera}
La atmósfera es una capa de gases que rodea a un planeta o a un cuerpo material, la cual se mantiene en su lugar deibido a la fuerza de gravedad del cuerpo.

Si la gravedad a la que se somete la atmósfera es alta y su temperatura es baja, es muy probable que dicha atmósfera se retenga.

La atmósfera ayuda a proteger a los organismos vivos de los daños que pueda ocacionar la radiación ultravioleta, rayos cosmicos y viento solar.

\section*{Composición de la atmósfera terrestre}
La atmósfera de la tierra está principalmente compuesta por un 78\% de nitrógeno, 21\% de oxigeno, 0.9\% de argón con dióxido de carbono y algunos otros gases. El oxigeno es usado por la mayoria de los organismos para la respiración, el nitrógeno se utiliza para producir amoníaco, utilizado en la construcción de nucleótidos y aminoácidos, y el dioxido de carbono es usado por las plantas para fotosíntesis.

\section*{Presión atmosférica}
La presión atmosférica es la fuerza por unidad de área que es aplicada perpendicularmente a la superficie por un gas circundante.

Es determinado por la fuerza gravitacional ejercida por el planeta tomando en cuenta la masa total de una columna de gas sobre un lugar.

En la tierra, la unidad de presión que es internacionalmente reconocida es \textbf{atm} la cual se define como \textbf{101.325KPa}, \textbf{760 Torr} o \textbf{14.696PSI} la cual es medida con barometro.

\section*{Estructura de la atmósfera}
La atmosfera está compuesta en 4 capas: troposfera, estratosfera, mesosfera y termosfera.



La troposfera es la capa más baja de la atmósfera. En esta capa es donde se da la vida y donde se experimenta el clima. La temperatura en ésta generalmete decrece con respecto a la altura.

La zona fronteriza de la troposfera cona estratosfera se llama tropopausa. Es el punto más alto donde el clima puede ocurrir. La altura de la troposfera varia dependiendo de la locación.

En la estratosfera, la temperatura aumenta con respecto a la altura. Esto se debe a que la capa de ozono está contenida en la estratosfera, como la capa de ozono recive todos los rayos ultravioleta por parte del sol, la temperatura asciende.

La mesosfera es la capa que está por encima de la estratosfera. La temperatura decrece con respecto a la altura. En esta capa hay proporciones de oxigeno y nitrógeno, similar a la troposfera, sólo que es menos la concentración de estas y aquí no ocurre el clima.

La termosfera es la capa más alta de la atmósfera. en esta capa la temperatura incrementa con respecto a la altura porque es afectada directamente por el sol.

\section*{Datos atmósfericos}
Se pueden tomar datos de la atmósfera ya sean, temperatura, presión atmosférica, humedad y velocidad del viento, con un weather ballon o sounding ballon, que lleva instrumentos para tomar dichos datos. Los datos pueden ser rastreados por radar o sistema de navegación.
 


\end{document}
